\section{Related work}
\label{sec:related-work}

\begin{itemize}
    \item Victors \ea proposed the Onion Name System~\cite{Victors2017a}.
    \item Punycode issues~\cite{Zheng2017a}.  Not a problem for onion services
        since they are all ASCII characters.
    \item Sai and Fink proposed a mnemonic system that maps 80-bit onion domains
        to sentences~\cite{Sai2012a}.  Their work is inspired by mnemonicode, a
        method to map binary data to words~\cite{mnemonicode}.
    \item OpenSSH uses ASCII ``drawings'' to illustrate
        fingerprints~\cite{Loss2009a}.
    \item Kadianakis discussed ways that would make next-generation onion
        domains easier to work with~\cite{Kadianakis2017a}.
    \item Filast{\`o} and Appelbaum proposed a petname system for Tor2Web,
        allowing onion service operators to register
        mnemonics~\cite{Filasto2011a}.
    \item Plasmoid showed how users can be tricked into accepting malicious
        fingerprints, exploiting that most people only look at the first and
        last couple of digits~\cite{Plasmoid2003a}.
    \item Tan \ea looked at how different fingerprint comparison mechanisms
        fare~\cite{Tan2017a}.
    \item Namecoin is technology built on top of Bitcoin that implements a
        generic key $\rightarrow$ value mapping mechanism.  It can serve as a
        DNS replacement, but suffers from poor usability~\cite{Kalodner2015a}.
    \item Hash visualization~\cite{Perrig1999a,Dhamija2005a}.
    \item Bonneau and Schechter showed how 56-bit secrets can be stored in
        ``human memory''~\cite{Bonneau2014a}.
    \item Susceptibility to phishing~\cite{Downs2006a,Sheng2010a}.
    \item Some work on collaboration practices~\cite{Forte2017a}.

    % 2007
    \item Clark, van Oorschot and Adams used cognitive walkthroughs to study
        how users install, configure, and run Tor~\cite{Clark2007a}.  The
        authors uncovered several usability hurdles such as jargon-laden
        documentation, confusing menus, and insufficient visual feedback.  As
        of May 2017, the study is ten years old---Tor Browser has since seen
        radical changes.

    % 2014
    \item Norcie \ea identified stop-points in the installation and use
        of the Tor Browser Bundle~\cite{Norcie2014a}.\footnote{The Tor Browser
        Bundle was later rebranded and is now known as Tor Browser.}  These
        stop-points represent places in a user interface that require action
        but are met with confusion by users.  Having identified these stop
        points, the authors then issued interface design recommendations and
        subsequently tested these recommendations in a user study.

    % 2015
    \item Motivated by Tor's anti-censorship components, Fifield \ea published
        a design to study the usability of Tor as a censorship circumvention
        tool~\cite{Fifield2015a}.  The authors plan to recruit hundreds of
        users to study how they use Tor's configuration wizard in an
        adversarial setting.  This effort made use of both qualitative and
        quantitative methods, Lee \ea~\cite{Lee2017a} studied the usability of
        Tor Launcher, the graphical configuration tool that allows users to
        configure Tor Browser.  Their results paint a bleak picture; 79\% of
        users' connection attempts in a simulated censored environment failed.
        However, the researchers' showed that their interface improvements
        resulted in less difficulties for users.

    % 2017
    \item Forte \ea studied the privacy practices of contributors to open
        collaboration projects~\cite{Forte2017a}.  The authors interviewed 23
        contributors to The Tor Project and Wikipedia to learn about how
        privacy concerns affect their contribution practices.  The study found
        that contributors worry about an array of threats, including
        surveillance, violence, harrassment, and loss of opporunity.

    % 2017
    \item Gallagher \ea conducted a series of semi-structured interviews to
        understand both why people use Tor and how they understand the
        technology~\cite{Gallagher2017a}.  The authors found that experts tend
        to have a network-centric view of Tor and tend to use it frequently over
        time while non-actors have a goal-oriented view and see Tor as a black
        box that provides a service.  Consequently, non-experts don't use Tor if
        they don't need its service.  Furthermore, non-experts tend to consider
        a single threat while the threat model of experts contains multiple
        actors.

\end{itemize}

We hope that our results generalize to other systems that use
self-authenticating names such as Freenet~\cite{Freenet} and Bitcoin addresses.
